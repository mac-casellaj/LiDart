\documentclass{article}

\usepackage{tabularx}
\usepackage{booktabs}

\title{Problem Statement and Goals\\\progname}

\author{\authname}

\date{}

%% Comments

\usepackage{color}

\newif\ifcomments\commentstrue %displays comments
%\newif\ifcomments\commentsfalse %so that comments do not display

\ifcomments
\newcommand{\authornote}[3]{\textcolor{#1}{[#3 ---#2]}}
\newcommand{\todo}[1]{\textcolor{red}{[TODO: #1]}}
\else
\newcommand{\authornote}[3]{}
\newcommand{\todo}[1]{}
\fi

\newcommand{\wss}[1]{\authornote{blue}{SS}{#1}} 
\newcommand{\plt}[1]{\authornote{magenta}{TPLT}{#1}} %For explanation of the template
\newcommand{\an}[1]{\authornote{cyan}{Author}{#1}}

%% Common Parts

\newcommand{\progname}{Mechatronics Engineering} % PUT YOUR PROGRAM NAME HERE
\newcommand{\authname}{Team 10, LiDart
\\ Jonathan Casella
\\ Karim Elmokattaf
\\ Michaela Schnull
\\ Neeraj Ahluwalia} % AUTHOR NAMES                  

\usepackage{hyperref}
    \hypersetup{colorlinks=true, linkcolor=blue, citecolor=blue, filecolor=blue,
                urlcolor=blue, unicode=false}
    \urlstyle{same}
                                


\begin{document}

\maketitle

\begin{table}[hp]
\caption{Revision History} \label{TblRevisionHistory}
\begin{tabularx}{\textwidth}{llX}
\toprule
\textbf{Date} & \textbf{Developer(s)} & \textbf{Change}\\
\midrule
26/Sep/2022 & Jonathan Casella & Initial Release\\
\bottomrule
\end{tabularx}
\end{table}

\section{Problem Statement}

3D scanning should be a low cost and user friendly technology, so that it can be used by the widest audience.
It should not require expensive or specialized hardware, nor should it require that the user have any domain specific knowledge.

\subsection{Proposed Solution}

LiDart aims to build a low cost, simple to use, WiFi connected 3D scanning robot.
LiDart plans to do this by using easily available, low cost LiDAR sensors and consumer
grade webcams alongside inexpensive location markers.
The user will interface with the robot through a desktop application that will
allow them to remotely drive the robot and perform 3D scans.

\subsection{Desktop Application Inputs and Outputs}

The desktop application will connect to the robot over WiFi.
While connected the user will be able to drive the robot using the keyboard,
the user will also be able to see the video from the webcam(s) aboard the
robot and a visualization of the LiDAR data. 

While driving the robot, the user will be able to perform 3D scans.
The output of these scans can then be exported from the
desktop application and saved as a Wavefront OBJ file.

\subsection{Robot Inputs and Outputs}

Aboard the robot will be one or more consumer webcams,
a low cost LiDAR module and other miscellaneous sensors
for the various moving mechanisms. Data from all of these inputs
will be both relayed to the desktop application and used by the
software onboard the robot for tasks like localization. The robot will also receive commands over WiFi from the desktop application
that will be used to remotely drive the robot and perform 3D scans.

The video feed from the webcam(s) will be used to detect inexpensive location markers that
will be scattered around the robot's environment. 

\subsection{Stakeholders}
Stakeholders in the project include all parties that would benefit from
low cost, easily accessable 3D scanning. Examples of groups that fit this
criteria include small visual effects studios, independent mechanical
designers, and manufacturing firms.

All of the afformentioned groups lack domain specific knowledge related to 
robotics and 3D scanning, as such a focus for this project will be to make the
3D scanning process as user friendly as possible. 

\subsection{Environment}
The robot is expected to operate indoors on flat surfaces.
The robot's environment is also expected to have been pre-populated with
inexpensive location markers.

The desktop application will be designed to run on a computer
running a desktop operating system with WiFi connectivity and
sufficiently powerful hardware. The robot software will run aboard the robot
on an embedded computer.

\section{Goals}
\subsection{Accurate 3D Scanning}

The produced 3D scans should be accurate within a given threshold.
This threshold may not be as small as commercial solutions, but should
be small enough that the results can be used for most tasks.

For example, the scans produced by LiDart's robot may not be accurate enough
for use by an engineer designing mission critical components, but should
be accurate enough for rapid prototyping.

\subsection{Low Cost Hardware}

The hardware used to build the robot should be low cost and easily accessable.
It should also be able to swap components with equivalent alternatives
without any major effort.

For example, it should be easy to replace our choice of consumer webcam with another
choice that meets similar specifications.

\subsection{Ease of Use}

Producing 3D using the LiDart robot should only require the user to drive
around their object of interest and instruct the robot to perform a scan.
A user with no knowledge of the inner workings of the system should be able
to operate it with only a brief explanation. 

The ease of use could be measured using a group of test subjects with no
prior knowledge of the system.

\section{Stretch Goals}
\subsection{Autonomous Scanning}

The operation of the robot as specified above involves the user remotely
piloting the robot. As this could be time consuming for the user, an additional
feature LiDart could implement is autonomous scanning.

While autonomously scanning, the robot would explore its environment without
any user intervention, scanning while it goes. The user could then export this
scan and import it to their software package of choice to select the part of
the environment that they wish to keep.

\subsection{Camera Based Scanning}

As specified above, the robot will have both a LiDAR module and one
or more consumer grade webcams. It is possible to remove the LiDAR module
and instead use only consumer grade webcams for both localization and 3D scanning.
This would allow LiDart to reduce cost by removing the LiDAR.


\end{document}